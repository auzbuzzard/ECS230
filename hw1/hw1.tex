%% ECS230 - Homework 1
%% Eric Kalosa-Kenyon

\documentclass[12pt,fleqn,leqno,letterpaper]{article}

\usepackage{amssymb}

\include{preamble}

\title{Homework 1}
\author{Eric Kalosa-Kenyon\\
\small{ECS230}\\
\small{UC Davis}\\
\small{\texttt{ekal@ucdavis.edu}}
}
\date{October 2017}

\begin{document}

% \setstretch{1.00}
\maketitle

% -- Table of Contents --
% (would go here)

% -- set document spacing --
% \setstretch{1.09}  % single line
% \setstretch{1.30}  % single wide-spaced
% \setstretch{1.50}  % one and a half spacing

% -- Import content here
% \input{s_introduction}

%% BEGIN MAIN

\section{Introduction}
This report contains the exposition of five different elementary properties of
real matrices nominally related to inverses and perturbation.

\section{Properties of simple matrices}
\begin{enumerate}
    \item Let $A,B \in \mathbb{R}^{n\times n}$ and assume $A^{-1}$ and $B^{-1}$
        both exist. The inverse of $AB$ is $(AB)^{-1} = B^{-1}A^{-1}$. To verify
        this, I check that left and right multiplication by the proposed inverse
        indeed maps $AB$ to the real identity matrix $I_{n\times n}$.\\
        $(AB)^{-1}AB = B^{-1}A^{-1}AB = B^{-1}IB = B^{-1}B = I$.\\
        $AB(AB)^{-1} = ABB^{-1}A^{-1} = AIA^{-1} = AA^{-1} = I$.\\
        Hence, $(AB)^{-1} = B^{-1}A^{-1}$ as suggested.

    \item Let $A\in\mathbb{R}^{n\times n}$. $Ax=0$ for every $x\in\mathbb{R}^n$
        if and only if $A=0$. The $\Leftarrow$ proof is trivial and will be
        oommitted. The $\Rightarrow$ proof will be presented with a contradiction
        argument, as follows.\\
        Suppose there is an element $a_{ij}$ of $A$ that is non-zero. Selecting
        an $x\in\mathbb{R}^n$ with non-zero $x_j$ and zeros in every other
        index, the product vector $Ax$ has an entry in the $j$th column equal to
        precisely $a_{ij}x_j\neq 0$. Hence, for any matrix $A$ with at least one
        non-zero element, one can find a subspace of $\mathbb{R}^n$ of
        cardinality at least $\aleph_1$ containing elements such as $x$
        described above that lie orthogonal to the null-space of $A$.

    \item Let $A\in\mathbb{R}^{n\times n}$ be strictly lower triangular. $A^n$,
        the repeated multiplication of $A$ by itself $n$ times, is always
        $A^n=0$. To verify this, I show that the first $i$ rows of $A^i$ are
        $0^\top$.\\
        By the definition of strict lower triangularity, the first row of
        $A^1=A$ is trivially $0^\top$. Let $A_j^\top$ denote the vector
        representation of the $j$th row of $A$. The first row of $A^2$ is
        trivially $0^\top$ because the first row of $A^1$ is already $0^\top$.
        It remains to show that the second row of $A^2$ is $0^\top$. To do so,
        notice that $A_2^\top = [a_{2,1}, 0, \dots, 0]$ and that each
        column vector of $A$ has a sequence of $0$s in its first indices with
        a length least one $0$ i.e. at least the first entry is $0$ in all of
        $A$'s column vectors. Denote these vectors $A_j^\star$. Notice that the
        aforementioned zero indices of the $A_j^\star$ vectors force
        $\langle A_2^\top, A_j^\star \rangle = 0$ for all $j\in\{1,\dots,n\}$.
        Hence, the second row in $A^2$ is $0$. The strong inductive step
        trivially employs the same argument as above for each inductive index
        and is omitted here for brevity.

    \item Let $u,v\in\mathbb{R}^n$ and $A=I_n+uv^\top\in\mathbb{R}^{n\times n}$
        The inverse of $A$ is $(I+uv^\top)^{-1} = \alpha(I-uv^\top)$. To verify
        this, as in the first problem, left and right multiplication to identity
        are checked, as follows.\\
        $(I+uv^\top)(I-\alpha uv^\top) =
        I + uv^\top - \alpha uv^\top - \alpha uv^\top uv^\top =
        I + (1 - \alpha(1 - u^\top v)) uv^\top$.
        To clarify the latest simplification,
        $uv^\top uv^\top = u(v^\top u)v^\top = u(u^\top v)v^\top =
        (u^\top v) uv^\top$. Let $c=u^\top v$.
        Notice that to fulfil the properties of the inverse, we must select an
        $\alpha$ so that the preceeding coefficient of $uv^\top$ i.e.
        $1-\alpha(1-c)$ is $0$. Such an $\alpha$ is
        $1/(1-c)$ which exists so long as $\langle u, v \rangle \neq
        0$.\\
        Checking right multiplication,
        $(I-\alpha uv^\top)(I+uv^\top) =
        I - \alpha uv^\top + uv^\top - \alpha c uv^\top =
        I + (1 - \alpha(1+c))uv^\top = I$ as desired.

    \item pr5
\end{enumerate}

%% END MAIN

% -- Bibliography (APA style)
% \bibliography{references}

\end{document}

%%
%% LaTeX Boilerplate (http://github.com/gbluma/latex-boilerplate/)
%%
