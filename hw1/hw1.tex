%% ECS230 - Homework 1
%% Eric Kalosa-Kenyon

\documentclass[12pt,fleqn,leqno,letterpaper]{article}

\usepackage{amssymb}

\include{preamble}

\title{Homework 1}
\author{Eric Kalosa-Kenyon\\
\small{ECS230}\\
\small{UC Davis}\\
\small{\texttt{ekal@ucdavis.edu}}
}
\date{October 2017}

\begin{document}

% \setstretch{1.00}
\maketitle

% -- Table of Contents --
% (would go here)

% -- set document spacing --
% \setstretch{1.09}  % single line
% \setstretch{1.30}  % single wide-spaced
% \setstretch{1.50}  % one and a half spacing

% -- Import content here
% \input{s_introduction}

%% BEGIN MAIN

\section{Introduction}
This report contains the exposition of five different elementary properties of
real matrices nominally related to inverses and perturbation.

\section{Properties of simple matrices}
\begin{enumerate}
    \item Let $A,B \in \mathbb{R}^{n\times n}$ and assume $A^{-1}$ and $B^{-1}$
        both exist. The inverse of $AB$ is $(AB)^{-1} = B^{-1}A^{-1}$. To verify
        this, I check that left and right multiplication by the proposed inverse
        indeed maps $AB$ to the real identity matrix $I_{n\times n}$.\\
        $(AB)^{-1}AB = B^{-1}A^{-1}AB = B^{-1}IB = B^{-1}B = I$.\\
        $AB(AB)^{-1} = ABB^{-1}A^{-1} = AIA^{-1} = AA^{-1} = I$.\\
        Hence, $(AB)^{-1} = B^{-1}A^{-1}$ as suggested.

    \item Let $A\in\mathbb{R}^{n\times n}$. $Ax=0$ for every $x\in\mathbb{R}^n$
        if and only if $A=0$. The $\Leftarrow$ proof is trivial and will be
        oommitted. The $\Rightarrow$ proof will be presented with a contradiction
        argument, as follows.\\
        Suppose there is an element $a_{ij}$ of $A$ that is non-zero. Selecting
        an $x\in\mathbb{R}^n$ with non-zero $x_j$ and zeros in every other
        index, the product vector $Ax$ has an entry in the $j$th column equal to
        precisely $a_{ij}x_j\neq 0$. Hence, for any matrix $A$ with at least one
        non-zero element, one can find a subspace of $\mathbb{R}^n$ of
        cardinality at least $\aleph_1$ containing elements such as $x$
        described above that lie orthogonal to the null-space of $A$.

    \item asdfh
\end{enumerate}

%% END MAIN

% -- Bibliography (APA style)
% \bibliography{references}

\end{document}

%%
%% LaTeX Boilerplate (http://github.com/gbluma/latex-boilerplate/)
%%
