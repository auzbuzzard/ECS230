%% ECS230 - Homework 2
%% Eric Kalosa-Kenyon

\documentclass[12pt,fleqn,leqno,letterpaper]{article}

\usepackage{amssymb}
\usepackage{amsmath}

\setlength{\parindent}{0cm} % Default is 15pt.

\include{preamble}

\title{Homework 2}
\author{Eric Kalosa-Kenyon\\
\small{ECS230}\\
\small{UC Davis}\\
\small{\texttt{ekal@ucdavis.edu}}
}
\date{October 2017}

\begin{document}

% \setstretch{1.00}
\maketitle

% -- Table of Contents --
% (would go here)

% -- set document spacing --
% \setstretch{1.09}  % single line
% \setstretch{1.30}  % single wide-spaced
% \setstretch{1.50}  % one and a half spacing

% -- Import content here
% \input{s_introduction}

%% BEGIN MAIN

\section{Introduction}
This report contains a relative error bound calculation for a simple `gaxpy`
problem and some performance measurements for different methods of square root
calculation.

\section{Relative error bounds}
Consider the solution to
$$Ax=b \textrm{ for } A\in\mathbb{R}_{2\times 2}
\textrm{ and } b\in\mathbb{R}^2, b\neq 0$$
given that the relative error on $b$ is bounded as follows:
$$\frac{\|\delta b\|}{\|b\|}<\epsilon \textrm{ with } \epsilon>0$$
where the norm used here is the infinity norm
$$\|\omega\|=\|\omega\|_\infty=
\begin{cases}
    \underset{i}{\max} x_i &\text{if } \omega = x \in \mathbb{R}^n\\
    \underset{\sum_{j=1}^n}{\max} |a_{ij}| &\text{if } \omega = A \in
    \mathbb{R}_{n\times n}\\
\end{cases}
$$
unless otherwise stated.

Here, I consider the matrix
$$
A =
\left[
\begin{array}{cc}
    a & a-1 \\
    a-1 & a-2
\end{array}
\right]
\textrm{ with } a\in R, a>2
$$

Notice that $|A|=1$ so $A$ is nonsingular and invertible - the inverse matrix
follows:
$$
A^{-1} = \left[ \begin{array}{cc}
    2-a & a-1 \\
    a-1 & -a
\end{array}\right]
$$

The condition number for a matrix is defined as $K(A) = \|A\| \|A^{-1}\|$. Here,
this quantity works out to $K(A)=2a-1$. Next, I'll work out a bound on the
relative error $\|\delta x\|/\|x\|$.

Let $A(x + \delta x) = b + \delta b$ as in the standard backwards error
calculation framework. Notice that, because $Ax=b$, the preceeding equation
reduces to $A\delta x = \delta b$. $A$ is invertible, so
$\delta x = A^{-1}\delta b$. Using a ubiquetous theorem and invoking the
assumption on the relative error bound for $b$,
$$
\frac{\|\delta x\|}{\|x\|} \le K(A) \frac{\|\delta b\|}{\|b\|} < (2a-1)\epsilon
\quad\blacksquare
$$

\section{Performance measurements}
\begin{enumerate}
    \item Compiled code using \texttt{"gcc -o timing1 timing1.c -lm"}

    \item Ran \texttt{"timing1 5000000"} ten times with the following results:\\
        $$
        \begin{array}{lccc}
            run & clocks & t cpu (s) & t real (s) \\
            1 & 16634 & 0.016634 & 0.016634 \\
            2 & 16578 & 0.016578 & 0.016578 \\
            3 & 16672 & 0.016672 & 0.016672 \\
            4 & 16608 & 0.016822 & 0.016822 \\
            5 & 16608 & 0.016608 & 0.016608 \\
            6 & 16199 & 0.016199 & 0.016200 \\
            7 & 16739 & 0.016739 & 0.016739 \\
            8 & 17086 & 0.017086 & 0.017087 \\
            9 & 16133 & 0.016133 & 0.016134 \\
            10 & 16758 & 0.016758 & 0.016759
        \end{array}
        $$
        The square root proceedure $\sqrt{x}$ on $x=5000000$ takes about
        $\aprox1650\pm 500$ clock cycles to converge. The clock cycles roughly
        equate to the number of microseconds required to compute the result, and
        these compute-time estimates are exactly as precise as the clock count.

    \item Ran \texttt{"timing1 10000000"} ten times with the following
        results:\\
        $$
        \begin{array}{lccc}
            run & clocks & t cpu (s) & t real (s) \\
            1 & 32790 & 0.032790 & 0.032791 \\
            2 & 33462 & 0.033462 & 0.033625 \\
            3 & 32733 & 0.032733 & 0.032733 \\
            4 & 33037 & 0.033037 & 0.033038 \\
            5 & 32971 & 0.032971 & 0.032972 \\
            6 & 33374 & 0.033374 & 0.033376 \\
            7 & 33649 & 0.033649 & 0.033651 \\
            8 & 32594 & 0.032594 & 0.032596 \\
            9 & 33511 & 0.033511 & 0.033513 \\
            10 & 34108 & 0.034108 & 0.034108
        \end{array}
        $$
        The precision of each timing metric is identical, and the number of
        clock cycles equate to the `t cpu` at a different order of magnitude.
        The real time is not exactly the number of clock cycles, and this
        measure of processing time deviates from the `t cpu` more than the same
        program did for the $x=5000000$ case.

    \item Ran \texttt{"timing1 20000000"} ten times with the following
        results:\\
        $$
        \begin{array}{lccc}
            run & clocks & t cpu (s) & t real (s) \\
            1 & 67884 & 0.067884 & 0.067888 \\
            2 & 64784 & 0.064784 & 0.064787 \\
            3 & 66242 & 0.066242 & 0.066245 \\
            4 & 68415 & 0.068415 & 0.068417 \\
            5 & 65403 & 0.065403 & 0.065406 \\
            6 & 65250 & 0.065250 & 0.065253 \\
            7 & 64236 & 0.064236 & 0.064238 \\
            8 & 65454 & 0.065454 & 0.065457 \\
            9 & 68738 & 0.068738 & 0.068742 \\
            10 & 67162 & 0.067162 & 0.067165
        \end{array}
        $$
        The trend of a larger deviation of the `t real` from the `t cpu`
        continues in the $x=20000000$ case. Notice that `t real` is larger than
        or equal to `t cpu` in all cases. This discrepancy may be accounted for
        by other processes in the instruction queue.

    \item Ran \texttt{"timing2 5000000"} ten times with the following
        results:\\
        $$
        \begin{array}{lcccc}
            run & sum clocks & sum & sqrt clocks & sqrt \\
            1 & 144 & 3.162278 & 64851877 & 7.453559e+09 \\
            2 & 144 & 3.162278 & 68189708 & 7.453559e+09 \\
            3 & 140 & 3.162278 & 65772986 & 7.453559e+09 \\
            4 & 140 & 3.162278 & 64225261 & 7.453559e+09 \\
            5 & 144 & 3.162278 & 66887858 & 7.453559e+09 \\
            6 & 140 & 3.162278 & 65289854 & 7.453559e+09 \\
            7 & 144 & 3.162278 & 64459362 & 7.453559e+09 \\
            8 & 140 & 3.162278 & 64458638 & 7.453559e+09 \\
            9 & 144 & 3.162278 & 64603720 & 7.453559e+09 \\
            10 & 140 & 3.162278 & 66835988 & 7.453559e+09
        \end{array}
        $$

    \item Ran \texttt{"timing2 10000000"} ten times with the following
        results:\\
        $$
        \begin{array}{lcccc}
            run & sum clocks & sum & sqrt clocks & sqrt \\
            1 & 140 & 3.162278 & 130993985 & 2.108185e+10 \\
            2 & 144 & 3.162278 & 128799883 & 2.108185e+10 \\
            3 & 144 & 3.162278 & 129375638 & 2.108185e+10 \\
            4 & 140 & 3.162278 & 129031288 & 2.108185e+10 \\
            5 & 140 & 3.162278 & 131155445 & 2.108185e+10 \\
            6 & 144 & 3.162278 & 130016120 & 2.108185e+10 \\
            7 & 140 & 3.162278 & 131313273 & 2.108185e+10 \\
            8 & 144 & 3.162278 & 131039483 & 2.108185e+10 \\
            9 & 148 & 3.162278 & 130911335 & 2.108185e+10 \\
            10 & 144 & 3.162278 & 131578104 & 2.108185e+10
        \end{array}
        $$

    \item Ran \texttt{"timing2 20000000"} ten times with the following
        results:\\
        $$
        \begin{array}{lcccc}
            run & sum clocks & sum & sqrt clocks & sqrt \\
            1 & 140 & 3.162278 & 256688153 & 5.962848e+10 \\
            2 & 140 & 3.162278 & 257840863 & 5.962848e+10 \\
            3 & 140 & 3.162278 & 259313408 & 5.962848e+10 \\
            4 & 140 & 3.162278 & 256437463 & 5.962848e+10 \\
            5 & 144 & 3.162278 & 260193269 & 5.962848e+10 \\
            6 & 140 & 3.162278 & 259175737 & 5.962848e+10 \\
            7 & 145 & 3.162278 & 258611655 & 5.962848e+10 \\
            8 & 140 & 3.162278 & 258817863 & 5.962848e+10 \\
            9 & 148 & 3.162278 & 258686040 & 5.962848e+10 \\
            10 & 140 & 3.162278 & 261196322 & 5.962848e+10
        \end{array}
        $$
\end{enumerate}

%% END MAIN

% -- Bibliography (APA style)
% \bibliography{references}

\end{document}

%%
%% LaTeX Boilerplate (http://github.com/gbluma/latex-boilerplate/)
%%
